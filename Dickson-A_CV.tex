\documentclass[margin,line]{res}
\usepackage{setspace}
\usepackage[dvips]{color}
\usepackage{fancyhdr,url}
\usepackage{pagecounting}

\makeatletter
%\renewcommand\@biblabel[1]{\textbullet}
\makeatother

\oddsidemargin -.5in
\evensidemargin -.5in
\textwidth=6.0in
\itemsep=0in
\parsep=0in
\definecolor{gray}{rgb}{0.4,0.4,0.4}
\definecolor{blue}{rgb}{0.2,0.2,0.9}

\newenvironment{list1}{
  \begin{list}{\ding{113}}{%
      \setlength{\itemsep}{0.05in}
      \setlength{\parsep}{0in} \setlength{\parskip}{0in}
      \setlength{\topsep}{0in} \setlength{\partopsep}{0in} 
      \setlength{\leftmargin}{0.17in}}}{\end{list}}
\newenvironment{list2}{
  \begin{list}{$\bullet$}{%
      \setlength{\itemsep}{0.03in}
      \setlength{\parsep}{0in} \setlength{\parskip}{0in}
      \setlength{\topsep}{0.05in} \setlength{\partopsep}{0in} 
      \setlength{\leftmargin}{0.2in}}}{\end{list}}

\renewcommand{\familydefault}{\sfdefault}

\fancypagestyle{plain}{
\fancyhf{}
\fancyfoot[C]{\textcolor{gray}{A. Dickson - Curriculum Vitae -- Page \thepage~}~~~~~~~~~~~~~~~~~~~~~~~}
\renewcommand{\headrulewidth}{0pt}
\renewcommand{\footrulewidth}{0pt}}
\pagestyle{plain}

\begin{document}

\name{\Large {\bf Alex Dickson} \vspace*{.1in}}

\begin{resume}
\section{\sc \textcolor{blue}{Contact Information}}
\vspace{.05in}
\begin{tabular}{@{}p{2in}p{4in}}
603 Wilson Rd             & {\it Voice:}  (517) 884-8985 \\            
Office 310B               & {\it Web:} \url{www.egr.msu.edu/~alexrd} \\         
East Lansing, MI 48824  & {\it E-mail:}  alexrd@msu.edu\\
USA & {\it Twitter:} \url{@DicksonLab}\\
\end{tabular}

\section{\sc \textcolor{blue}{ Goals}}
{\bf{\emph{To apply novel tools and techniques from molecular simulation to biological systems relevant to human disease.}}}

\section{\sc \textcolor{blue}{ Specific Areas \\ of Interest}}
Kinetics-oriented drug discovery, molecular property prediction, algorithm development, biological systems modeling, network analysis,
non-equilibrium systems, machine learning

\section{\sc \textcolor{blue}{ Education and Experience}}
{\bf Michigan State University}, East Lansing, MI\\
\vspace {-0.05in}
\begin{list1}
\item[] {\bf Associate Professor}, Department of Biochemistry and Molecular Biology, July 2021--Present
\item[] {\bf Assistant Professor}, Department of Biochemistry and Molecular Biology, September 2015--July 2021
\begin{list2}
\item {Joint appointment with the Department of Computational Mathematics, Science and Engineering}
\end{list2}
\end{list1}

{\bf University of Michigan}, Ann Arbor, MI\\
\vspace {-0.05in}
\begin{list1}
\item[] {\bf Postdoctoral Fellow}, Department of Chemistry, September 2011--August 2015
\item[] Primary Mentor:  Charles L. Brooks, III
\begin{list2}
\item {Studied relative path probabilities to elucidate mechanisms of chaperone activity in \textit{E. coli}}
\item {Developed analysis techniques (``hub scores'') to elucidate high-level behaviors in complex networks}
\item {Created and applied new custom enhanced sampling methods to explore new areas of configuration space in high-dimensional order parameter spaces (``WExplore'')}
\item {Elucidated mechanisms of interaction for the intrinsically disordered chaperone HdeA}
\end{list2}
\end{list1}

{\bf University of Chicago}, Chicago, IL\\
\vspace {-0.05in}
\begin{list1}
\item[] {{\bf Ph.D.}, Department of Chemistry, July 2011}
\item[] {Dissertation Topic:  \textit{``Enhanced sampling methods for nonequilibrium systems''}}
\item[] {Advisor:  Aaron R. Dinner}
\begin{list2}
\item {Developed enhanced sampling methods to study systems driven out of equilibrium (``nonequilibrium umbrella sampling'')}
\item {Ran massively parallel simulations on supercomputing architectures}
\item {Studied dynamical phase transitions using large deviation theory}
\end{list2}
\item[] {{\bf M.S.}, Chemistry,  May 2007}
\end{list1}

{\bf University of Toronto}, Ontario, Canada\\
\vspace {-0.05in}
\begin{list1}
\item[] {\bf Hon. B. Sc.}, Chemical Physics (Minor: Mathematics),  May 2006
\item[] {Thesis:  \textit{``The effect of trajectory accuracy on statistical distributions in chaotic systems''}}
\end{list1}

\section{\sc \textcolor{blue}{ Personal Information}}
I am a dual Canadian-American citizen.\\
I am an ACS member 30919152, and a member of ACS COMP since 2015.

\newpage
\section{\sc \textcolor{blue}{Publications in submission}}

%\setlength\bibitemsep{1.5\itemsep}

\emph{[*] = corresponding author}
\vspace {0.05in}

\begingroup
\begin{spacing}{1.2}
\renewcommand{\section}[2]{}%
\begin{thebibliography}{10}
\providecommand{\url}[1]{\texttt{#1}}
\providecommand{\urlprefix}{URL }
\expandafter\ifx\csname urlstyle\endcsname\relax
  \providecommand{\doi}[1]{doi:\discretionary{}{}{}#1}\else
  \providecommand{\doi}{doi:\discretionary{}{}{}\begingroup
  \urlstyle{rm}\Url}\fi
\providecommand{\bibAnnoteFile}[1]{%
  \IfFileExists{#1}{\begin{quotation}\noindent\textsc{Key:} #1\\
  \textsc{Annotation:}\ \input{#1}\end{quotation}}{}}
\providecommand{\bibAnnote}[2]{%
  \begin{quotation}\noindent\textsc{Key:} #1\\
  \textsc{Annotation:}\ #2\end{quotation}}
\providecommand{\eprint}[2][]{\url{#2}}

\setlength{\itemsep}{0.15in}

\bibitem{Raeburn2021}
  Raeburn CB, Ormsby AR, Moily NS, Cox D, Ebbinghaus S, {\bf Dickson A}, McColl G and Hatters DM* (2021) A biosensor to gauge protein homeostasis resilience differences in the nucleus compared to cytosol of mammalian cells.
  \newblock \textit{bioRxiv} doi:https://doi.org/10.1101/2021.04.19.440383.

\end{thebibliography}
\end{spacing}
\endgroup

\section{\sc \textcolor{blue}{Independent Publications}}

%\setlength\bibitemsep{1.5\itemsep}

\emph{[*] = corresponding author}
\vspace {0.05in}

\begingroup
\begin{spacing}{1.2}
\renewcommand{\section}[2]{}%
\begin{thebibliography}{10}
\providecommand{\url}[1]{\texttt{#1}}
\providecommand{\urlprefix}{URL }
\expandafter\ifx\csname urlstyle\endcsname\relax
  \providecommand{\doi}[1]{doi:\discretionary{}{}{}#1}\else
  \providecommand{\doi}{doi:\discretionary{}{}{}\begingroup
  \urlstyle{rm}\Url}\fi
\providecommand{\bibAnnoteFile}[1]{%
  \IfFileExists{#1}{\begin{quotation}\noindent\textsc{Key:} #1\\
  \textsc{Annotation:}\ \input{#1}\end{quotation}}{}}
\providecommand{\bibAnnote}[2]{%
  \begin{quotation}\noindent\textsc{Key:} #1\\
  \textsc{Annotation:}\ #2\end{quotation}}
\providecommand{\eprint}[2][]{\url{#2}}

\setlength{\itemsep}{0.15in}
\bibitem{Donyapour2021_sampl7}
Donyapour N and {\bf Dickson A}* (2021) Predicting partition coecients for the SAMPL7 physical property challenge using the ClassicalGSG method.
\newblock \textit{J. Comp. Aided Drug Design} 35:819-830.

\bibitem{Uyar2021}
Uyar A and {\bf Dickson A}* (2021) Perturbation of ACE2 structural ensembles by SARS-CoV-2 spike protein binding.
\newblock \textit{J. Chem. Theory Comput.} In press.

\bibitem{Donyapour2021}
Donyapour N, Hirn, MJ, {\bf Dickson A}* (2021) ClassicalGSG: Prediction of logP Using Classical Molecular Force Fields and Geometric Scattering for Graphs.
\newblock \textit{J. Comput. Chem.} 42:1006-1017.

\bibitem{Dixon2020}
Dixon T, Uyar A, Ferguson-Miller S, {\bf Dickson A}* (2020) Membrane-mediated ligand unbinding of the PK-11195 ligand from the translocator protein (TSPO).
\newblock \textit{Biophys. J.}, 120:158-167.

\bibitem{Lotz2020}
Lotz SD and {\bf Dickson A}* (2020) Wepy: A Flexible Software Framework for Simulating Rare Events with Weighted Ensemble Resampling.
\newblock \textit{ACS Omega}, 5:31608-31623.

\bibitem{Roussey2020}
Roussey NM and {\bf Dickson A}* (2020) Enhanced Jarzynski Free Energy Calculations using Weighted Ensemble.
\newblock \textit{J. Chem. Phys.} 153:134116.

\bibitem{Hall2020}
Hall R, Dixon T, {\bf Dickson A}* (2020) On Calculating Free Energy Using Ensembles of Transition Paths. 
\newblock \textit{Frontiers in Molecular Biosciences} 7:106.

\bibitem{Rizzi2020}
Rizzi A, Jensen T, Slochower DR, Aldeghi M, Gapsys V, Ntekoumes D, Bosisio S, Papadourakis M, Henriksen NM, de Groot BL, Cournia Z, {\bf Dickson A}, Michel J, Gilson MK, Shirts MR, Mobley DM* and Chodera JD* (2020) The SAMPL6 SAMPLing challenge: assessing the reliability and efficiency of binding free energy calculations. 
\newblock \textit{Journal of Computer Aided Molecular Design} 34:601-633.

\bibitem{Bogetti2019}
  Bogetti AT, Mostofian B, {\bf Dickson A}, Pratt AJ, Saglam AS, Harrison PO, Adelman JL, Dudek M, Torrillo PA, DeGrave AJ, Adhikari U, Zwier MC, Zuckerman DM* and Chong LT* (2019) A Suite of Tutorials for the WESTPA Rare-Events Sampling Software [Article v1.0].
\newblock \textit{Living Journal of Computational Molecular Science} 1(2):10607.

\bibitem{Donyapour2019}
  Donyapour N, Roussey NM and {\bf Dickson A*} (2019) REVO: Resampling of Ensembles by Variation Optimization.
  \newblock \textit{Journal of Chemical Physics} 150:244112.

\bibitem{Liu2019}
  Liu Y, Hickey DP, Minteer SD, {\bf Dickson A*} and Barton SC* (2019) Markov-State Transition Path Analysis of Electrostatic Channeling.
  \newblock \textit{Journal of Physical Chemistry C} 123: 15284-15292. {\bf **Cover article**}

\bibitem{Kirberger2019}
  Kirberger SE, Ycas PD, Johnson JA, Chen C, Ciccone M, Lu RWW, Urick AK, Zahid H, Shi K, Aihara H, McAllister SD, Kashani-Sabet M, Shi J, {\bf Dickson A}, dos Santos CO* and Pomerantz WCK* (2019) Selectivity, ligand deconstruction, and cellular activity analysis of a BPTF bromodomain inhibitor.
  \newblock \textit{Organic \& Biomolecular Chemistry} 17: 2020-2027.

\bibitem{Bai2019}
  Bai N, Roder H, {\bf Dickson A} and Karanicolas J* (2019) Isothermal Analysis of ThermoFluor Data can readily provide Quantitative Binding Affinities.
  \newblock \textit{Scientific Reports} 9:2650.

\bibitem{Dickson2018}
  {\bf Dickson A*} (2018) Mapping the Ligand Binding Landscape.
  \newblock \textit{Biophysical Journal} 115:1707-1719.

\bibitem{Dixon2018}
  Dixon T, Lotz SD, {\bf Dickson A*} (2018)
  Predicting Ligand Binding Affinity Using On- and Off-Rates for the SAMPL6 SAMPLing Challenge.
  \newblock \textit{Journal of Computer-Aided Molecular Design} 32(10):1001-1012 {\bf **Cover article**}

\bibitem{Zeng2018}
  Zeng X, Uyar A, Sui D, Donyapour N, Wu D, {\bf Dickson A*}, Hu J* (2018)
  Structural Insights Into Lethal Contractural Syndrome Type 3 (LCCS3) Caused by a Missense Mutation of PIP5Ky.
  \newblock \textit{Biochemical Journal} 475: 2257-2269.

\bibitem{Lotz2017}
  Lotz SD and {\bf Dickson A*} (2018)
  Unbiased Molecular Dynamics of 11 min Timescale Drug Unbinding Reveals Transition State Stabilizing Interactions.
\newblock \textit{Journal of the American Chemical Society} 140: 618-628.

\bibitem{Wood2017}
  Wood RJ, Ormsby AR, Radwan M, Cox D, Sharma A, Vopel T, Ebbinghaus S, Oliveberg M, Reid GE, {\bf Dickson A}, Hatters DM* (2018)
  A Biosensor-Based Framework to Measure Latent Proteostasis Capacity.
\newblock \textit{Nature Communications} 9:287.

\bibitem{Uyar2017}
Uyar A, Karamyan VT and {\bf Dickson A*} (2018)
Long-Range Changes in Neurolysin Dynamics Upon Inhibitor Binding.
\newblock \textit{Journal of Chemical Theory and Computation} 14: 444-452.

\bibitem{Dickson2017_review}
  {\bf Dickson A*}, Tiwary P and Vashisth H (2017)
Kinetics of Ligand Binding Through Advanced Computational Approaches: A Review.
\newblock \textit{Current Topics in Medicinal Chemistry} 17: 2626-2641.

\bibitem{Dickon2016_tryp}
{\bf Dickson A*}, Lotz SD (2017)
Multiple Unbinding Pathways and Ligand-Induced Destabilization Revealed by WExplore and Conformation Space Networks.
\newblock \textit{Biophysical Journal} 112: 620-629. 

\bibitem{Dickon2016_csa}
{\bf Dickson A*}, Bailey CT and Karanicolas J (2016)
Optimal Allosteric Stabilization Sites Using Contact Stabilization Analysis.
\newblock \textit{Journal of Computational Chemistry} 38: 1138-1146.

\bibitem{Dickson2016}
{\bf Dickson A*}, Lotz SD (2016)
Ligand Release Pathways Obtained with WExplore: Residence Times and Mechanisms.
\newblock \textit{Journal of Physical Chemistry B} 120: 5377-5385. 
\bibAnnoteFile{Dickson2016}

\end{thebibliography}
\end{spacing}
\endgroup

\section{\sc \textcolor{blue}{Publications with Former Mentors}}

%\setlength\bibitemsep{1.5\itemsep}

\emph{[*] = corresponding author}
\vspace {0.05in}

\begingroup
\begin{spacing}{1.2}
\renewcommand{\section}[2]{}%
\begin{thebibliography}{10}
\providecommand{\url}[1]{\texttt{#1}}
\providecommand{\urlprefix}{URL }
\expandafter\ifx\csname urlstyle\endcsname\relax
  \providecommand{\doi}[1]{doi:\discretionary{}{}{}#1}\else
  \providecommand{\doi}{doi:\discretionary{}{}{}\begingroup
  \urlstyle{rm}\Url}\fi
\providecommand{\bibAnnoteFile}[1]{%
  \IfFileExists{#1}{\begin{quotation}\noindent\textsc{Key:} #1\\
  \textsc{Annotation:}\ \input{#1}\end{quotation}}{}}
\providecommand{\bibAnnote}[2]{%
  \begin{quotation}\noindent\textsc{Key:} #1\\
  \textsc{Annotation:}\ #2\end{quotation}}
\providecommand{\eprint}[2][]{\url{#2}}

\setlength{\itemsep}{0.15in}

\bibitem{Salmon2016}
Salmon L, Ahlstrom LS, Horowitz S, {\bf Dickson A}, Brooks~{III} CL*, Bardwell, JCA* (2016)
Capturing a dynamic chaperone-substrate interaction using NMR-informed molecular modeling.
\newblock \textit{Journal of the American Chemical Society} 138: 9826-9839.
\bibAnnoteFile{Salmon2016}

\bibitem{Dickson_WEUS}
{\bf Dickson A}, Ahlstrom LS, Brooks~{III} CL* (2015)
Coupled Folding and Binding with 2D Window-Exchange Umbrella Sampling.
\newblock \textit{Journal of Computational Chemistry} 37: 587-594. {\bf **Cover article**}
\bibAnnoteFile{Dickson_WEUS}

\bibitem{Ahlstrom2015}
Ahlstrom LS, Law S, {\bf Dickson A}, Brooks~{III} CL* (2015)
Multiscale Modeling of a Conditionally Disordered {pH}-Sensing Chaperone
\newblock \textit{Journal of Molecular Biology} 427: 1670-1680. 
\bibAnnoteFile{Ahlstrom2015}

\bibitem{Laricheva2015}
Laricheva E, Goh G, {\bf Dickson A}, Brooks~{III} CL* (2015)
{pH}-Dependent Transient Conformational States Control Optical Properties in Cyan Fluorescent Protein
\newblock \textit{Journal of the American Chemical Society} 137: 2892-2900. 
\bibAnnoteFile{Laricheva2015}

\bibitem{Dickson_TAR}
{\bf Dickson A}, Mustoe AM, Salmon L, Brooks~{III} CL* (2014)
 Efficient In-Silico Exploration of {RNA} Interhelical Conformations Using Euler Angles and {WE}xplore.
\newblock \textit{Nucleic Acids Research} 42: 97-110.
\bibAnnoteFile{Dickson_TAR}

\bibitem{Dickson_WExplore}
{\bf Dickson A}, Brooks~{III} CL* (2014) {WE}xplore: Hierarchical
  exploration of high dimensional spaces using the weighted ensemble algorithm.
\newblock \textit{Journal of Physical Chemistry B} 118: 3532-3542. {\bf **Cover article**}
\bibAnnoteFile{Dickson_WExplore}

\bibitem{Dickson2013_foldeco}
{\bf Dickson A}, Brooks~{III} CL* (2013) Quantifying chaperone-mediated
  transitions in the proteostasis network of {E}. coli.
\newblock \textit{PLoS Computational Biology} 9: e1003324.
\bibAnnoteFile{Dickson2013_foldeco}

\bibitem{Dickson2013_hubshaw}
{\bf Dickson A}, Brooks~{III} CL* (2013) {Native states of fast-folding proteins are
  kinetic traps.}
\newblock \textit{Journal of the American Chemical Society} 135: 4729--4734.
\bibAnnoteFile{Dickson2013_hubshaw}

\bibitem{Ahlstrom2013}
  Ahlstrom LS, {\bf Dickson A}, Brooks~{III} CL* (2013)
  Binding and Folding of the Small Bacterial Chaperone HdeA.
\newblock \textit{Journal of Physical Chemistry B} 117: 13219-13225.
\bibAnnoteFile{Ahlstrom2013}

\bibitem{Dickson2012_hub}
{\bf Dickson A}, Brooks~{III} CL* (2012) Quantifying hub-like behavior in protein
  folding networks.
\newblock \textit{Journal of Chemical Theory and Computation} 8: 3044--3052.
\bibAnnoteFile{Dickson2012_hub}

\bibitem{Dickson2011_ldf}
{\bf Dickson A}, Tabei SMA, Dinner AR* (2011) {Entrainment of a driven oscillator as a
  dynamical phase transition}.
\newblock \textit{Physical Review E} 84: 061134.
\bibAnnoteFile{Dickson2011_ldf}

\bibitem{Dickson2011_RNA}
{\bf Dickson A}, Maienschein-Cline M, Tovo-Dwyer A, Hammond JR, Dinner AR* (2011)
  Flow-dependent unfolding and refolding of an {RNA} by nonequilibrium umbrella
  sampling.
\newblock \textit{Journal of Chemical Theory and Computation} 7: 2710--2720.
\bibAnnoteFile{Dickson2011_RNA}

\bibitem{Dickson2010_shear}
{\bf Dickson A}, Nasto A, Dinner AR* (2010) {Incorporating friction and collective
  shear moves into a lattice gas.}
\newblock \textit{Physical Review E} 81: 051111.
\bibAnnoteFile{Dickson2010_shear}

\bibitem{Dickson2010_review}
{\bf Dickson A}, Dinner AR* (2010) Enhanced sampling of nonequilibrium steady states.
\newblock \textit{Annual Reviews of Physical Chemistry} 61: 441--459.
\bibAnnoteFile{Dickson2010_review}

\bibitem{Dickson2009_rate}
{\bf Dickson A}, Warmflash A, Dinner AR* (2009) Separating forward and backward
  pathways in nonequilibrium umbrella sampling.
\newblock \textit{Journal of Chemical Physics} 131: 154104.
\bibAnnoteFile{Dickson2009_rate}

\bibitem{Dickson2009}
{\bf Dickson A}, Warmflash A, Dinner AR* (2009) Nonequilibrium umbrella sampling in
  spaces of many order parameters.
\newblock \textit{Journal of Chemical Physics} 130: 074104.
\bibAnnoteFile{Dickson2009}

\end{thebibliography}
\end{spacing}
\endgroup

\section{\sc \textcolor{blue}{Funding} }

{\bf NIH NIGMS} R01 Administrative Supplement \hfill {\bf 2019} Total: \$124,464 USD \\
\emph{Mapping the ligand binding landscape with advanced molecular simulation methods}\\
{\bf PI:} Alex Dickson (MSU)

{\bf NIH NIGMS} R01 Grant \hfill {\bf 2018-2022} Total: \$1.3 M USD \\
\emph{Mapping the ligand binding landscape with advanced molecular simulation methods}\\
{\bf PI:} Alex Dickson (MSU)

{\bf DMS/NIGMS} Research at the Interface of Bio. and Math. Sciences \hfill {\bf 2018-2022} Total: \$1.1 M USD \\
 \emph{Kinetics-driven drug discovery using persistent homology, rare event molecular dynamics and experimental data}\\
 {\bf Lead PI:} Alex Dickson (MSU), {\bf Co-PIs:} Guowei Wei (MSU), Kin Sing Stephen Lee (MSU)\\

{\bf Human Frontiers Science Program}, Program Grant \hfill {\bf 2017-2020} Total: \$1.2 M USD \\
 \emph{Defining the Capacity of Cells to Keep the Proteome Folded Over Space and Time}\\
 {\bf Lead PI:} Danny Hatters (U. Melbourne), {\bf Co-PIs:} Alex Dickson (MSU), Hannah Nicholas (U. Sydney), Simon Ebbinghaus (Ruhr. Univ. Bochum) \\

\section{\sc \textcolor{blue}{ Articles \\ Reviewed}}
I have served as a referee for articles submitted to the journals \textit{Proceedings of the National Academy of Sciences}, \textit{Journal of the American Chemical Society}, \textit{Nature Communications}, \textit{Biophysical Journal}, \textit{Journal of Chemical Information and Modeling}, \textit{Journal of Molecular Graphics and Modelling}, \textit{Journal of Computational Chemistry}, \textit{Biopolymers}, \textit{Computational Biology and Chemistry}, \textit{Frontiers in Bioengineering and Biotechnology}, \textit{PLoS Computational Biology}, \textit{Scientific Reports}, \textit{Journal of Physical Chemistry B} and \textit{Proteins: Structure, Function and Bioinformatics}.

\section{\sc \textcolor{blue}{ Proposals \\ Reviewed}}
{\bf Panel Member}, NIH \hfill 2021/10 ZRG1 BCMB-G (02) M, 2021. \\
{\bf Ad hoc Reviewer}, NSF CAREER \hfill Chemical Theory, Models and Computational Methods, 2020. \\
{\bf Panel Member}, NSF \hfill DMS NIGMS-A, 2019.\\
{\bf Ad hoc Reviewer}, NSF CAREER \hfill Chemistry of Life Processes, 2019. \\
{\bf Ad hoc Reviewer}, NSF CAREER \hfill Chemistry of Life Processes, 2019. \\
{\bf Ad hoc Reviewer}, NSF CAREER \hfill Chemistry of Life Processes, 2018. \\
{\bf Ad hoc Reviewer}, NSF CAREER \hfill Chemical Theory, Models and Computational Methods 2018. \\

\section{\sc \textcolor{blue}{ Teaching }}
{\bf CMSE201} \hfill {\bf Fall 2021} \\
\emph{Introduction to Computational Modeling}, 60 lecture hours, Lead Instructor managing 6 sections.\\
An introductory undergraduate course that teaches Python programming and computer modeling of real-world systems. Solo-taught one 70-student section.

{\bf BMB803/805} \hfill {\bf Spring 2021} \\
\emph{Protein Structure, Design and Mechanism}, 11 lecture hours.\\
A series of lectures and computational laboratories on protein sequence, structure and function.  Developed a final project component where students create molecular visualizations to present a molecular system to a broad audience.

{\bf BMB803/805} \hfill {\bf Spring 2020} \\
\emph{Protein Structure, Design and Mechanism}, 11 lecture hours.\\

{\bf CMSE201} \hfill {\bf Fall 2019} \\
\emph{Introduction to Computational Modeling}, 60 lecture hours.\\
An introductory undergraduate course that teaches Python programming and computer modeling of real-world systems. Solo-taught one 70-student section.

{\bf BMB803/805} \hfill {\bf Spring 2019} \\
\emph{Protein Structure, Design and Mechanism}, 11 lecture hours.\\

{\bf CMSE201} \hfill {\bf Fall 2018} \\
\emph{Introduction to Computational Modeling}, 60 lecture hours.\\
Solo-taught one 70-student section.

{\bf CMSE201} \hfill {\bf Fall 2017} \\
\emph{Introduction to Computational Modeling}, 6 lecture hours.\\
Taught three sections of a guest lecture on Langevin dynamics.

{\bf BMB961/CMSE890} \hfill {\bf Spring 2017} \\
\emph{Concepts in Protein Structure and Modeling}, 12 lecture hours.\\
Lectures and laboratories concerning enhanced sampling methods, clustering and kinetic modeling, and binding free energy calculations. Co-taught with Dr. Michael Feig.

{\bf CMSE801} \hfill {\bf Fall 2016} \\
\emph{Introduction to Computational Modeling}, 24 lecture hours.\\
Co-taught with Dr. Alexei Bazavov.
Graduate course with lectures and laboratories on creating physical and computational models; programming in Python; effective visualization of data; working in teams to computationally solve scientific problems.

{\bf NSC801} \hfill {\bf Spring 2016} \\
\emph{Introduction to Computational Modeling}, 24 lecture hours.\\
Graduate course co-taught with Dr. Mohsen Zayernouri.
All course material was developed from scratch by myself and Dr. Zayernouri.

\section{\sc \textcolor{blue}{ Mentees }}

{\bf Current}\\
Dr. Samik Bose (Postdoctoral researcher) \\
Tom Dixon (Ph.D. candidate) \\
Nicole Roussey (Ph.D. candidate) \\
Nazanin Donyapour (Ph.D. candidate) \\
James Lennon (Ph.D.) \\

{\bf Past}\\
Dr. Indrajit Deb (Postdoctoral researcher), now at PharmCADD \\
Dr. Arzu Uyar (Postdoctoral researcher) \\
Samuel D. Lotz (Ph.D. Biochemistry \& Molecular Biology, 2021), now at Roivant Sciences.\\
Thomas Diaz (M.Sc. Chemistry, 2019)\\
Robert Hall (Post baccalaureate student)\\

\section{\sc \textcolor{blue}{ Invited Talks }}

{\bf MolSSI School on Open-Source Software for Rare-Event Sampling Strategies}, June 2021 (virtual).\\
{\bf Biophysics Seminar}, University of Maryland College Park, April 2021 (virtual).\\
{\bf The Future of AI in Drug Discovery}, ACS Spring 2021 (virtual).\\
{\bf Biochemistry and Molecular Biology Seminar}, University of Oklahoma Health Sciences Center, March 2021 (virtual).\\
{\bf Physical Chemistry Seminar}, University of Florida, February 2021 (virtual).\\
{\bf Biophysics \& Molecular Biology Seminar}, Washington University in St. Louis, January 2021 (virtual).\\
{\bf Chemistry Colloquium}, University of Chicago, January 2021 (virtual).\\
{\bf Workshop on Free Energy Methods in Drug Design}, Novartis, Cambridge, November 2020 (virtual).\\
{\bf Integrated Applied Mathematics Seminar}, University of New Hampshire, November 2020 (virtual).\\
{\bf Departmental Seminar}, Biochemistry \& Molecular Biology, Michigan State University, October 2020 (virtual).\\
{\bf Online Presentation}, ACS Virtual Meeting, Fall 2020.\\
{\bf Departmental Seminar}, Chemical and Biomolecular Engineering, University of California, Irvine, May 2020 (virtual) \\
{\bf Departmental Seminar}, Biochemistry \& Molecular Biology, University of Oklahoma, April 2020 (cancelled due to COVID-19) \\
{\bf From protein folding to ligand binding: adventures with free energies}, ACS Meeting, March 2020 (cancelled due to COVID-19) \\
{\bf Biophysics Seminar}, University of Michigan, September 2019.\\
{\bf Free energy calculations: Entering the fourth decade of adventure in chemistry and biophysics}, Santa Fe, June 2019.\\
{\bf Sampling conformations and pathways in biomolecular systems}, ACS Meeting, April 2019.\\
{\bf Bioinformatics Cluster Meeting}, University of Rochester, April 2019.\\
{\bf Physics Seminar}, Oakland University, December 2018.\\
{\bf Chemistry Seminar}, William Paterson University, October 2018.\\
{\bf WExplore lecture @ WESTPA Workshop}, University of Pittsburgh, August 2018.\\
{\bf WExplore plugin tutorial @ WESTPA Workshop}, University of Pittsburgh, August 2018.\\
{\bf Mid-Atlantic Comp Chem Meeting} (via web) with Merck, GlaxoSmithKline, Bristol-Myers Squibb, and Pfizer, March 2018. \\
{\bf Seminar}, Department of Computational \& Systems Biology, University of Pittsburgh, September 2017. \\
{\bf Modeling \& Measuring Protein-Ligand Kinetics \& Residence Times}, ACS Meeting, August 2017. \\
{\bf Physical Chemistry Seminar}, Michigan State University, September 2016. \\
{\bf Charlie Brooks 60th Birthday Celebration}, ACS Meeting, March 2016. \\
{\bf Pharmacology \& Toxicology Seminar}, Michigan State University, February 2016. \\
{\bf Computer Science and Engineering Seminar}, Michigan State University, October 2015. \\
{\bf WESTPA Workshop}, University of Pittsburgh, July 2015. \\
{\bf University of British Columbia, Okanagan Campus}, Department of Chemistry, February 2015. \\
{\bf Michigan State University}, Department of Biochemistry and Molecular Biology, February 2015. \\
{\bf IUPUI}, Department of Chemistry, February 2015. \\
{\bf University of Texas, El Paso}, Department of Chemistry, December 2014. \\
{\bf University at Buffalo}, Department of Chemistry, November 2014. \\
{\bf Protein Folding Consortium} in Ann Arbor, MI, June 2014. \\
{\bf Washington University St. Louis}, Department of Biochemistry and Molecular Biophysics, March 2014. \\
{\bf Protein Folding Consortium Workshop} in Berkeley, June 2013. \\
{\bf After Mini Mini} for path sampling methods in Berkeley, Jan 2010. \\
{\bf Multiscale Materials Modeling} in Tallahassee, Oct 2008. \\

\section{\sc \textcolor{blue}{ Contributed Talks and \\ Poster Sessions}}

{\bf ACS Meeting} (Poster), San Diego, August 2019. \\
{\bf Biophysical Society Meeting} (Talk), Baltimore, March 2019.\\
{\bf Free Energy Methods, Kinetics and Markov State Models in Drug Design} (Poster), Cambridge, May 2018. \\
{\bf Binding Kinetics: Time is of the Essence} (Poster), Berlin, October 2017.\\
{\bf ACS Meeting} (Talk), Washington DC, August 2017. \\
{\bf ACS Meeting} (Talk), San Francisco, April 2017. \\
{\bf Biophysical Society Meeting} (Talk), Los Angeles, February 2016. \\
{\bf q-bio Winter Meeting} in Hawaii (Poster), February 2014. \\

%\vspace*{-2.5mm}

%\newpage

\section{\sc \textcolor{blue}{ Honors and Awards} }
{\bf OpenEye Outstanding Junior Faculty Award}, American Chemical Society, COMP Division, 2019. \\
{\bf Elizabeth R. Norton Prize for Excellence in Research in Chemistry}, University of Chicago, 2010. \\
{\bf Postgraduate Scholarship}, Natural Sciences and Engineering Research Council of Canada (NSERC), 2007-2011.\\
{\bf Freud Scholar Fellowship}, University of Chicago, 2006. \\
{\bf Undergraduate Student Research Award}, NSERC, 2005. \\
{\bf Undergraduate Student Research Award}, NSERC, 2004. \\
{\bf Susan C. Gollop and William G. Gollop Scholarship in Chemistry}, University of Toronto, 2002-2003. \\

\section{\sc \textcolor{blue}{Service}}
{\bf Co-chair}, COMP Programming Committee, American Chemical Society, 2020-Present.\\
{\bf Co-organizer}, Biophysical Society Virtual Networking Event: Biomolecular Modeling in the Age of Machine Learning, May 2021.\\
{\bf Organizer/presider}, American Chemical Society Meeting: Kinetics of Macromolecular Systems, Virtual, Spring 2021.\\
{\bf Organizer/presider}, American Chemical Society Meeting: The Future of AI in Chemistry and Drug Discovery, Virtual, Spring 2021.\\
{\bf Presider}, American Chemical Society Meeting: Sampling Conformations \& Pathways in Biomolecular Systems, Orlando, April 2019.\\
{\bf Session chair}, Biophysical Society Meeting: Protein-Small Molecular Interactions, Baltimore, March 2019.\\
{\bf Organizer}, Midwest Computational Chemistry Conference, East Lansing, MI, June 2017. \\
{\bf Organizer}, BMB Departmental Retreat, East Lansing, MI, September 2016. \\
{\bf Organizer}, Computational Biology Symposium, Ann Arbor, MI, September 2016. \\
{\bf Session chair}, Biophysical Society Meeting: Protein-Small Molecule Interactions, Los Angeles, February 2016. \\
{\bf Session chair}, Gordon Research Seminar: Computational Chemistry, July 2014. \\

\section{\sc \textcolor{blue}{Community Outreach}}
        {\bf MSU Science Festival}, Abrams Planetarium \hfill {\bf 2016-2019} \\
        \emph{Small Molecules on the Big Screen} (2019)\\
        \emph{Small Molecules on the Big Screen} (2018)\\
        \emph{Small Molecules on the Big Screen} (2017)\\
        \emph{What Happens to Molecules When You Turn Up the Heat?} (2016)\\
        40-min talks with dome-formatted molecular visualizations designed to appeal to younger audiences.  Topics include: phase transitions (ice $\rightarrow$ water $\rightarrow$ steam), DNA synthesis, destabilizing protein mutations, molecular definition of temperature.

        {\bf MSU Grandparents' University} \hfill {\bf 2016-2017} \\
        \emph{How Do Molecules Find Their Perfect Fit?}\\
        Taught two 1.5 hr lectures for children and their grandparents, employing 3D printed models of proteins to demonstrate key principles of molecular interaction.  Topics included: assembly of viral capsids, molecular recognition, protein conformational change.

\section{\sc \textcolor{blue}{ Departmental Service}}
        
        {\bf Organizing committee}, Science at the Edge, Michigan State University \hfill {\bf 2019-Present}\\
        {\emph {Member}}

        {\bf Computer committee}, BMB, Michigan State University \hfill {\bf 2018-Present}\\
        {\emph {Chair}}

        {\bf Undergraduate Studies Committee}, CMSE, Michigan State University \hfill {\bf 2017-Present}\\
        {\emph {Member}}

        {\bf Faculty Advisory Committee}, BMB, Michigan State University \hfill {\bf 2019-2021}\\
        {\emph {Member}}

        {\bf Faculty Advisory Committee}, BMB, Michigan State University \hfill {\bf 2018}\\
        {\emph {Ad Hoc Member}}

        {\bf Search Committee, Fixed Term Assistant Professor}, Mathematics, Michigan State University \hfill {\bf 2018}\\
        {\emph {Member}}
        
        {\bf Computer Committee}, BMB, Michigan State University \hfill {\bf 2015-2017}\\
        {\emph {Member}}

        {\bf Committee on Undergraduate Minor}, CMSE, Michigan State University \hfill {\bf 2015-2017}\\
        {\emph {Member}}

{\bf Tiger Talks}, Department of Chemistry, University of Chicago

\vspace{-.3cm}
{\em Co-chair, Graphic Designer} \hfill {\bf 2009-2010}\\
Ran a series of talks featuring senior graduate students presenting to an audience of their peers.

{\bf PSD Social Committee}, Physical Sciences Division, University of Chicago

\vspace{-.3cm}
{\em Beer Baron} \hfill {\bf 2007-2009}\\
Helped organize many social events for graduate students in the PSD, including a weekly happy
hour. Responsibilities included sampling and purchasing craft beers from local breweries.

{\bf Student Seminar Series}, Department of Chemistry, University of Chicago

\vspace{-.3cm}
{\em Co-creator, Student Interviewer} \hfill {\bf 2007}\\
The goal of this seminar series was to provide junior graduate students with an intimate window into
the lives of accomplished academics in chemistry. Two live, one-on-one interviews were conducted
with Professors Harry Gray and Fraser Stoddart in front of an audience of graduate students.

{\bf Chemistry Students' Union}, Department of Chemistry, University of Toronto

\vspace{-.3cm}
{\em President} \hfill {\bf 2005-2006}\\
Along with a small group of students from my cohort, I transformed the CSU from a defunct
organization to a vibrant, active team of students that is still in operation today. The CSU organizes
social and academic events for undergraduate
students in the Department of Chemistry.

{\bf First-year Learning Communities}, University of Toronto

\vspace{-.3cm}
{\em Student Mentor} \hfill {\bf 2005-2006}\\
Led a weekly course that helped integrate first-year students into undergraduate life.
The course was designed to teach academic and personal skills that help students succeed in a University environment.
Responsibilities included developing a curriculum, making lesson plans, and leading a class of 24 students.

\end{resume}
\end{document}

